%% start of file `template.tex'.
%% Copyright 2006-2010 Xavier Danaux (xdanaux@gmail.com).
%% Copyright 2010-2011 Mark Liu (markwayneliu@gmail.com).
%
% This work may be distributed and/or modified under the
% conditions of the LaTeX Project Public License version 1.3c,
% available at http://www.latex-project.org/lppl/.

\documentclass[11pt,a4paper,sans]{moderncv}

\usepackage{verbatim}

% moderncv themes
\moderncvstyle{classic}
\moderncvcolor{blue}

% character encoding
\usepackage[utf8]{inputenc}                   % replace by the encoding you are using

% adjust the page margins
\usepackage[scale=0.8]{geometry}
%\setlength{\hintscolumnwidth}{3cm}                     % if you want to change the width of the column with the dates
%\AtBeginDocument{\setlength{\maketitlenamewidth}{6cm}}  % only for the classic theme, if you want to change the width of your name placeholder (to leave more space for your address details
%\AtBeginDocument{\recomputelengths}                     % required when changes are made to page layout lengths


% personal data
\firstname{Catalin}
\familyname{Moraru}
%\address{(omitted for web)}{(omitted for web)}    % optional, remove the line if not wanted
\email{catalin.me@gmail.com}                      % optional, remove the line if not wanted
\mobile{(+40727320330)}                    % optional, remove the line if not wanted
%\homepage{http://markliu.me}                % optional, remove the line if not wanted
%\extrainfo{\url{http://markliu.me}} % optional, remove the line if not wanted

% to show numerical labels in the bibliography; only useful if you make citations in your resume
%\makeatletter
%\renewcommand*{\bibliographyitemlabel}{\@biblabel{\arabic{enumiv}}}
%\makeatother

%\nopagenumbers{}                             % uncomment to suppress automatic page numbering for CVs longer than one page
%----------------------------------------------------------------------------------
%            content
%----------------------------------------------------------------------------------
\begin{document}
\maketitle

%% Not relevant right now
%\section{General information}
%\cventry{Full name}{Catalin Eugen Moraru}{}{}{}{}
%\cventry{Date of birth}{23.02.1987}
%\cventry{Address}{Str. Laborator nr 145, Sector 3 Bucharest, Romania}
%\cventry{Phone}{+40 727320330}
%\cventry{E-mail}{catalin.me@gmail.com}
%

%\cventry{}{}{}{}{}{}

\section{Education} %--------------------------------------------------

\cventry{2010--2011}{Master Degree}{University POLITEHNICA of Bucharest}{}{\newline{}Automatic Control and Computers Faculty}{Area of Study: Distributed Systems\newline{}Dissertation title:\textit{Scalable Overlay for P2P Social Networks}}
%GPA 10.00

\cventry{2006--2010}{Bachelor Degree}{University POLITEHNICA of Bucharest}{}{\newline{}Automatic Control and Computers Faculty}{Area of study: Computer Science}
%GPA 10.00

\cventry{2002--2006}{Baccalaureate}{Mihai Viteazul National College}{\textit{Ploiesti}}{}{Area of study: Mathematics - Informatics}
%GPA 9.70


\section{Professional Experience} %-------------------------------------
%\cventry{year--year}{Job title}{Employer}{City}{}{Description}

\cventry{2015--Present}{Senior Software Engineer}{OTC Intel Romania}{Bucharest}{}{
\begin{itemize}
    \item {DSLTE Hybrid Access}
        \begin{itemize}
            \item The project aims to create a bandwidth aggregation solution for DSL and LTE WAN access links. The solution is a packet-switched aggregation that benefits from custom MPTCP congestion control to deal with the very different link technologies
            \item The skill-set involved Linux TCP stack understanding, advanced performance tuning and debugging at both system and networking level.
            \item Part of the team that developed the project from POC to product state.
            \item Sustained a demo of the project at Mobile World Congress 2016 as part of the Intel Demo team
        \end{itemize}
\end{itemize}
}


\cventry{2013--2015}{Software Engineer}{OTC Intel Romania -- as a RINF Contingent Worker}{Bucharest}{}{
\begin{itemize}
    \item{AppMigrator}
        \begin{itemize}
            \item Part of a 3 member team that designed and implemented a solution for migrating running applications between Android mobile devices. Applications were virtualized by means of linux-based containers. The migration was seamless not only in terms of downtime, but also in terms of network responsiveness.
            \item Based on the different types of applications, different live migration strategies were used (mainly a mix of post-copy and pre-copy migration) 
            \item Development areas include: virtualization, live-migration, KVM, Multipath TCP, kernel drivers, Qemu C++ development
            \item The project is part of a research EU FP7 Consortium with multiple industry and academia partners
            \item The project was awarded R\&D Project of the year 2016 by ANIS
        \end{itemize}
    \item {Multipath TCP}
        \begin{itemize}
            \item Involved in testing and benchmarking the MPTCP open-source implementation from user-space point of view through applications like VM Migration \item Involved in patchwork for the existing opens-source implementation and setup of various test environments for MPTCP. A spinoff of this project was the integration of MPTCP for redundancy with an Openstack-controlled cluster deployment as a joined project Intel Cloud Platform Division.
        \end{itemize}
\end{itemize}
}

\cventry{}{}{}{}{}{
\begin{itemize}
    \item {Android on ChromeOS}
        \begin{itemize}
            \item Developed a POC of a running Android OS on ChromeOS platform by integrating Qemu with NativeClient. The end result is a way to have virtualized Android applications on ChromeOS. The POC was then later on handed to a product team for further development.
            \item The work resulted from the POC was accepted as a presentation at Intel OSTS 2014
        \end{itemize}  
    \item {SGX enabling on Openstack}
        \begin{itemize}
            \item Worked with the Intel SGX Romania division to allow the Openstack framework to take advantage of the Intel SGX extension
            \item My role involved configuring the Openstack framework in order to better fit an SGX-capable Virtual Machine
        \end{itemize}
\end{itemize}
}


\cventry{2011--2013}{C\textbackslash C++ Developer}{Axway Romania}{Bucharest}{}{
\begin{itemize}
    \item Project: Integrator: highly scalable, “any-to-any” and “many-to-many” transformation engine that offers full support for standard and non-standard document types, direct database read/write, and multiple input, target and reference files in a single transformation routine. Tasks included development, maintenance and continuous integration, remote support sessions on client production environment regarding functional and performance issues
    \item Project: B2BI: Business to Business Integration application that enables high data availability and consistency. Tasks included maintenance and optimization of a key core component of B2BI, as well as continuing to offer remote support for clients
    \item During this period I improved my debugging skills as a result of debugging real world distributed systems on product environments in time critical situations. 
    \item I have also increased my ability to internalize new tools and development systems.
    \item Technologies used: 
    \begin{itemize}
        \item Languages:Java, C, Python, MessageBuilder (Axway proprietary)
        \item Development platforms: Linux, Windows, Solaris, HPUX, AIX 
        \item Tools: Mantise, svn, git, Eclipse IDE, Intellij IDEA, OpenGrok + ctags, Linux env tools, WinDdg, (gdb, strace, netstat, etc.), OracleDB, SQL db, Jenkins, VMWare Workstation, Axway specific monitoring and debugging tools
    \end{itemize}
\end{itemize}
}

\cventry{2010--2012}{Associate Teaching Assistant}{University POLITEHNICA of Bucharest}{\newline{}Automatic Control and Computers Faculty}{}{
{Course:} Operating Systems \newline{}%
{Activities:}%
\begin{itemize}
    \item Developed laboratory resources for bachelor students in the 3rd year of study
    \item Assisted the laboratory classes and guided students towards a better understanding of the theoretical concepts through hands-on tasks
    \item Helped students to solve basic system programming tasks using both Linux and Windows specific APIs
 \end{itemize}
}

\cventry{2010--2011}{C\textbackslash C++ embedded Developer}{Virtual Metrix}{Bucharest}{}{
\begin{itemize}
    \item Project: VMXL4 
    \item Part of a 4 member team that enabled a paravirtualized Linux to run on top of a L4 family microkernel. The project was targeted for ARM platforms: Beagleboard ARM v6 and Pandaboard ARM v7
    \item During this project I have improved my knowledge of Operating System internals and ARM assembly, and also increased my knowledge of the C language
    \item Technologies used: git, C/C++,  various linux tools (mkfs, objdump, readelf, gdb, uboot), BusyBox, Skyeye simulator, Codesourcery toolchain, Gnu ARM toolchain, Scons python, Wiki, Trac
\end{itemize}
}

\cventry{2008--2009}{Junior embedded C Developer}{Luxoft Romania}{Bucharest}{}{
\begin{itemize}
    \item Project: Baystack - OS for layer 2/3 switches products for Nortel
    \item Involved in maintanance tasks with the following areas from the project:  QoS, LDAP, LLDP, SNMP
    \item Technologies used: C, Wind River Workbench, ClearCase
\end{itemize}
}

\section{Patents Co-author}
\cventry{Approved for filing in 2015}{Mobile device navigation method for finding points of interest}{Intel ID:122552}{}{}{}
\cventry{Filed in 2014}{Mobile device navigation method for management of large groups}{Intel ID:117892}{}{}{}

\section{Certificates}
\cventry{2010}{Advanced Computer Science and Business Oracle Academy Program}{\newline{}University POLITEHNICA of Bucharest}{\newline{}Certificate of course completion}{}{}
\cventry{2009}{Cisco Certified Network Associate}{AcademiaCisco, UPB}{\newline{}Certificate of course completion}{}{}


\section{Computer skills}
\cvitem{Area of expertise}{\begin{itemize}
\item Deep understanding of client/server programming, systems programming, distributed system applications optimized for performance and scalability
\item Strong Linux experience, configuration and administration skills, knowledge of Operating System internals
\item Strong knowledge of generic algorithms and distributed algorithms
\end{itemize}}
\cvitem{Programming Languages}{C, C++, Python, Java, C\#, OpenMP, MPI, OpenGL, Bash Scripting}
\cvitem{Operating Systems}{UNIX based systems (Linux, HPUX, AIX), Windows, Android, WindRiver}
\cvitem{DB Management}{MySQL, Oracle}


%\section{Awards and Grants}


\section{Languages}
\cvitemwithcomment{Romanian}{Speaking 5, Writing 5}{Native language}
\cvitemwithcomment{English}{Speaking 4, Writing 3}{}
\cvitemwithcomment{German}{Speaking 2, Writing 2}{}
\cvitemwithcomment{}{}{Marked 1-5, where 5 is excellent}

\end{document}

